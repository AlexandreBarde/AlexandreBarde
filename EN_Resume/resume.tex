\documentclass[
	%a4paper, % Uncomment for A4 paper size (default is US letter)
	11pt, % Default font size, can use 10pt, 11pt or 12pt
]{resume} 

\usepackage{geometry} 

%------------------------------------------------

\name{Barde Alexandre} 

\address{Software and data engineer \\ Specialized in the space industry}

\address{Available in Montreal in September}

\address{alexandre@brde.fr \\ https://alexandre.brde.fr} 

%----------------------------------------------------------------------------------------

\begin{document}

%----------------------------------------------------------------------------------------
%	EDUCATION SECTION
%----------------------------------------------------------------------------------------

\begin{rSection}{About me}

	I am a software engineer who has worked on a variety of projects, all of which have space in common. 
	I am used to working on projects with large data streams. I also had the opportunity to work on many different languages, those where I practice the most are Java, Python and Golang. 
	I also work daily on cloud technologies, such as Kubernetes, Helm and Docker. All this is orchestrated on automated pipelines via different tools such as GitlabCI.

\end{rSection}

\begin{rSection}{Education}
	
	\textbf{Master’s degree in computer science} \hfill \textit{2019 - 2021} \\ 
	Double degree : \\
	- IT expert and information system \\
	- Software development expert \\
	Creation of Android applications in Java, whether for management applications or virtual reality games. 
	Machine learning and deep learning for processing massive data and creating a chatbot to help diagnose a covid case.
	Creation of pipelines and scripts on GitlabCI to learn how to set up development and integration automation.
	\hfill { YNOV Campus, Toulouse }


	\textbf{Bachelor degree in software development and quality} \hfill {\em 2018 - 2019} \\ 
	Advanced Java skills and various web languages, such as PHP, NodeJS and client-server architectures. 
	Mastery of the main agile software development methods, such as Scrum and eXtreme Programming, Test Driven Development.
	\hfill { Paul Sabatier, Toulouse }


	\textbf{Associate degree in computer science} \hfill {\em 2016 - 2018} \\
	Learning of several languages, like C, Java, Python, ADA. Design and administration of databases with Oracle or MySQL. 
	Advanced use of Linux on a daily basis, writing different scripts as well as learning about the network and the administration and maintenance of Linux servers.
	\hfill { Paul Sabatier, Toulouse }

	\textbf{Technological baccalaureate STISD} \hfill {\em 2015 - 2016} \\ 
	Sciences and Technologies of Industry and Sustainable Development - 
	Information System and Digital Specialty Study of digital processing, learning of electronics and development of embedded software. \\ 
	Design of a project to convert a street lamp into a connected street lamp, in order to be able to control it remotely via a web application and to automate it. 
	Live recovery of the values of the batteries and the solar panels.
	\hfill { Lycée Georges Cabanis, Brive La Gaillarde }

\end{rSection}

%----------------------------------------------------------------------------------------
%	WORK EXPERIENCE SECTION
%----------------------------------------------------------------------------------------

\begin{rSection}{Professional experience}

	\begin{rSubsection}{Software and data engineer on Galileo}{November 2021 - Now}{Thales Alenia Space, Toulouse}{}
		\item Galileo is the European satellite positioning system initiated by the European Union. 
		Creation of a tool that allows to retrieve data from the constellation, to group them and to process them.
 		\item Use of Java, as well as SpringBoot for the management of microservices.
 		\item Deploy using cloud technologies like Kubernetes, Helm and Docker.
 		\item Using RabbitMQ and its clients in different languages to manage the data flow.
		\item \textbf{Technologies used :} Java, Python, Golang, Kubernetes, Helm, Docker, GitlabCI, Minio, RabbitMQ, Redis, InfluxDB.
	\end{rSubsection}

	\begin{rSubsection}{Full stack developer - Apprenticeship}{October 2019 - October 2021}{Thales Alenia Space, Toulouse}{}
		\item G2G is the program for the renewal of the Galileo satellite constellation via an ESA call for tender. 
			Development of prototypes for the planning management of all the activities of the constellation and the ground segment. 
			Like the maintenance of a satellite or the contact from the ground to a satellite to send navigation data.
		\item \textbf{Technologies used :} Golang, MongoDB, Docker, NodeJS, Swagger, tests end to end, GitlabCI.
		\item Second project: Architecture, development and cloud deployment of a monitoring and predictive maintenance 
			product for satellite constellation monitoring (HUMS: Health and Usage Monitoring System).
		\item \textbf{Technologies used :} Kubernetes, Helm, Docker, Python, PostgreSQL, Kafka, nagular, NodeJS, Azure.
	\end{rSubsection}

	\begin{rSubsection}{Full stack developer - Internship}{April 2019 - August 2019}{Thales Alenia Space, France}{}
		\item Development of a Web GUI to supervise and configure satellite telecommunications network equipment installed in a traffic docking station.
		\item \textbf{Technologies used :} Angular, Grafana, API Rest and WebSocket.
	\end{rSubsection}

	\begin{rSubsection}{Web Developer - Internship}{April 2018 - June 2018}{GNH Conseil, Toulouse}{}
		\item Development of a Web tool to promote the implementation of actions in favor of sustainable development.
		\item \textbf{Technologies used :} Object-oriented PHP, MySQL database, Javascript.
	\end{rSubsection}

\end{rSection}

%----------------------------------------------------------------------------------------
%	TECHNICAL STRENGTHS SECTION
%----------------------------------------------------------------------------------------

\begin{rSection}{Technical skills}

	\begin{tabular}{ @{} >{\bfseries}l @{\hspace{6ex}} l }
		Software languages & C, C++, C\#, Golang, Java\\
		Languages & Python, Django, Bash, PHP, JavaScript, NodeJS  \\
		Technologies & MVC, HTML5, CSS, Latex, GIT\\
		Databases - NoSQL & MongoDB, Cassandra, Redis\\
		Databses - SQL & MariaDB, MySQL, SQLite, Oracle, PostgreSQL, InfluxDB\\
		Tests & TDD, Junit, PHPUnit, Robot Framework\\
		Cloud & Azure, Kubernetes, Helm, Docker, Firebase, Azure\\
		CI/CD & GitlabCI, Jenkins
	\end{tabular}

\end{rSection}

\begin{rSection}{Projects}
	{\bf TeamMatesFinder - YNOV Campus}
	\\ Platform that allows you to search for video game teammates based on specific criteria such as their level, their role or if they are sociable.
	\\ \textbf{Technologies used :} Docker, PostgreSQL, Angular, ExpressJS, NodeJS.

	{\bf Online store - AtelierCausseNature}
	\\ Development of an online store of a craftsman who works with wood to produce jewelry or decorative objects.
	\\ \textbf{Technologies used :} NGINX, Symfony, PHP, MySQL, Bootstrap.

	{\bf DevOPS pipeline - YNOV Campus}
	\\ Automated deployment via the GitLab CI. Run tests, build, create a Docker image for deployment, error handling with Sentry and notification system via Slack.
	\\ \textbf{Technologies used : }GitLab CI, Docker, Shell, API Slack.

	{\bf Android Development - YNOV Campus}
	\\ Development of an Android application to manage a child’s schedule, with tasks, alarm clocks and reminders. 
	It allows to leave autonomy to the child so that it is responsible. This application is written in Java via Android Studio.
	\\ \textbf{Technologies used :} Java - Android Studio, Firebase.

	{\bf Technical referent database - Aldostra}
	\\ Advice and management of the databases of the Aldostra project, which is a Minecraft server on the Harry Potter universe, several hundreds of players 
	connect to it in order to embody a character. I also set up the migration of our databases from Redis to MySQL by creating Java scripts to transfer the thousands of data.
	\\ \textbf{Technologies used :} Java, Redis, MySQL.

	{\bf Co-manager of the 1st French-speaking forum Skript - Skript-MC}
	\\ Skrip-MC is the first French-speaking Skript forum, which is a Minecraft plugin allowing to simplify the development on the game. 
	Our community has nearly 25,000 members, I manage the team and the technical aspects of the forum.
	\\ \textbf{Technologies used :} PHP, NGINX, XenForo, communication on social networks, team management.

	{\bf DreamTech - Bachelor degree in software development and quality}
	\\ Development of a collaborative and playful platform to connect students wishing to revise. 
	It allows to find partners to revise or an experienced student in a subject who seeks to help others. 
	It integrates a gamification system to motivate users to use the site and its features, such as a system of experience, rewards or ranking. 
	Implemented a continuous integration system with TravisCI and PHPUnit tests to guarantee a good maintainability of the project.
	\\ \textbf{Technologies used :} Symfony, TravisCI, Bootstrap, JavaScript, MySQL, PHPUnit, Scrum.


\end{rSection}

\begin{rSection}{Certifications}

    \begin{rSubsection}{Electrical authorization - B0 H0 BR BE Essai}{INRS France}{}{}
	\item B0 : Possibility of doing non-electrical work on low voltage equipment
	\item H0 : Ability to do non-electrical work on high voltage equipment
	\item BR : In charge of general interventions on low voltage installations
	\item BE Essai : Specific operations (tests, verification, measurement, operation) on low voltage installations
    \end{rSubsection}

    \begin{rSubsection}{Certified Kubernetes Application Developer (CKAD) - In progress}{Cloud Native Computing Foundation - The Linux Foundation}{}{}
	\item Core concepts : Understand how the Kubernetes API primitives work. Create and configure Pods.
	\item Configuration : Understand how ConfigMaps and SecurityContexts work. Definition of the resources of an application. Create and use Secrets. Understand how ServiceAccounts work.
	\item Multi-Container Pods : Understand the multi-container architecture of Pods.
	\item Observabiliy : Understanding how LivenessProbes and ReadinessProbes work. Handling container logs. Application monitoring in Kubernetes. Debugging in Kubernetes.
	\item Pod Design : Deployments, RollingUpdate and rollbacks. Understanding how Jobs and CronJobs work. Handling Labels, Selectors and Annotations.
	\item Services and Networking : Understand how the Services work. Handling NetworkPolicies.
	\item State Persistence : Use of PersistentVolumeClaims for data storage.
    \end{rSubsection}

\end{rSection}

\end{document}
