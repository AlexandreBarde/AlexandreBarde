\documentclass[
	%a4paper, % Uncomment for A4 paper size (default is US letter)
	10pt, % Default font size, can use 10pt, 11pt or 12pt
]{resume} % Use the resume class

\usepackage{geometry} % Use the EB Garamond font

%------------------------------------------------

\name{Barde Alexandre}

\address{Ingénieur logiciel spécialisé dans le développement logiciel cloud-native et le domaine spatial}

\contact{alexandre.brde@gmail.com \\ https://alexandre.brde.fr \\ +1 (438) 338 3007}

\address{Montréal, H3H 0B2 }

%----------------------------------------------------------------------------------------

\begin{document}

%----------------------------------------------------------------------------------------
%	EDUCATION SECTION
%----------------------------------------------------------------------------------------

\begin{rSection}{Parcours scolaire}
	
	\textbf{Mastère informatique} \hfill \textit{2019 - 2021} \\ 
	Double diplôme : \\
	- Expert informatique et système d'information \\ 
	- Expert en développement logiciel \\
	Création d'applications Android en Java, Javascript et Unity, pour des applications de gestion ou des jeux en réalité virtuelle. 
	Apprentissage du machine learning et deep learning pour le traitement de données massives et la création d'un chatbot pour aider à diagnostiquer un cas covid. 
	Création de pipelines et de scripts sur GitlabCI pour la mise en place d'automatisation de développement, d'intrégration et déploiement.
	\hfill { YNOV Campus, Toulouse }

	\textbf{Licence professionnelle développement et qualité des logiciels} \hfill {\em 2018 - 2019} \\ 
	Développement en Java avancée et de différents langages web, tel que le PHP, NodeJS ainsi que des architectures clients serveurs. 
	Maitrise des principales méthodes agiles de développement logiciel, comme le Scrum et l'eXtreme Programming, le Test Driven Development. 
	\hfill { Paul Sabatier, Toulouse }

	\textbf{IUT Informatique} \hfill {\em 2016 - 2018} \\
	Apprentissage de plusieurs langages, comme le C, le Java, le Python, le PHP, l'ADA. Conception et administration de base de données avec Oracle ou MySQL. 
	Utilisation avancée de Linux au quotidien, écriture de différents scripts ainsi que l'apprentissage du réseau et l'administration et le maintient de serveur Linux.
	\hfill { Paul Sabatier, Toulouse }

	{\bf Baccalauréat technologique STI2D } \hfill {\em 2015 - 2016} \\ 
	Sciences et Technologies de l'Industrie et du Développement Durable - Spécialité Système d'Information et Numérique
	Étude du traitement numérique, apprentissage de l'électronique ainsi que du développement de logiciel embarqué. 
	Conception d'un projet pour convertir un lampadaire en lampadaire connecté, afin de pouvoir le piloter à distance via une application web et de l'automatiser. 
	\hfill { Lycée Georges Cabanis, Brive La Gaillarde }

\end{rSection}

%----------------------------------------------------------------------------------------
%	WORK EXPERIENCE SECTION
%----------------------------------------------------------------------------------------

\begin{rSection}{Experience professionnelle}

	\begin{rSubsection}{Ingénieur logiciel / data et Scrum Master sur Galileo}{Novembre 2021 - Maintenant}{Thales Alenia Space, Toulouse}{}
		\item Animation et préparation des différentes cérémonies dans un projet SAFe.
		\item Développement de microservices JAVA avec SpringBoot afin de récupérer différents flux (MDDN, OSPF, SIS, PRS, etc) du système de positionnement par satellites européeen en temps réel, de les décoder, traiter et socker dans des base de données. Afin de monitorer l'état de santé de la constellation. 
 		\item Déploiement en utilisant des technologies cloud comme Kubernetes, Helm et Docker.
 		\item Utilisation de RabbitMQ et de ses clients dans différents langages pour gérer le flux de données. Redis pour du Caching et InfluxDB pour du stockage.
		\item \textbf{Technologies utilisées : }Java, Python, Golang, Kubernetes, Helm, Docker, GitlabCI, Minio, RabbitMQ, Redis, InfluxDB.
	\end{rSubsection}

	\begin{rSubsection}{Développeur full stack - Alternance}{Octobre 2019 - Octobre 2021}{Thales Alenia Space, Toulouse}{}
 		\item Développement de l'outil de gestion de planning de toutes les activités de la constellation et du segment sol du projet Galileo Second Generation. Comme la maintenance d'un satellite ou le contact du sol vers un satellite pour l'envoi de données de navigation.
 		\item Second projet de l'alternance : Architecture, développement et déploiement cloud d'un produit de monitoring et de maintenance prédictive 
			pour la surveillance de constellation de satellites (HUMS : Health and Usage Monitoring System).
		\item \textbf{Technologies utilisées :} Golang, NodeJS, Python, C++, MongoDB, PostgreSQL, Kafka, Docker, Helm, Kubernetes, Azure, Angular, Swagger, GitlabCI.
	\end{rSubsection}

	\begin{rSubsection}{Développeur full stack - Stage}{Avril 2019 - Août 2019}{Thales Alenia Space, France}{}
 		\item Développement d'un outil Web afin de superviser et configurer des équipements du réseau de télécommunications du satellite installés dans une station d'ancrage du traffic.
		\item  \textbf{Technologies utilisées :} Angular, Grafana, API Rest et WebSocket.
	\end{rSubsection}

	\begin{rSubsection}{Développeur full stack - Stage}{Avril 2018 - Juin 2018}{GNH Conseil, Toulouse}{}
 		\item Architecture, développement et déploiement d'un outil Web favorisant la mise en oeuvre d'actions en faveur du développement durable.
		\item \textbf{Technologies utilisées :} PHP, MySQL, Bootstrap, Javascript, NGINX.
	\end{rSubsection}

\end{rSection}

%----------------------------------------------------------------------------------------
%	TECHNICAL STRENGTHS SECTION
%----------------------------------------------------------------------------------------

\begin{rSection}{Compétences techniques}

	\begin{tabular}{ @{} >{\bfseries}l @{\hspace{6ex}} l }
		Langages & C, C++, C\#, Golang, Java, Python, Bash, PHP, JavaScript, LateX  \\
		Base de données - NoSQL & MongoDB, Cassandra, Redis, Firebase\\
		Base de données - SQL & MariaDB, MySQL, SQLite, Oracle, PostgreSQL, InfluxDB\\
		Tests & TDD, Junit, PHPUnit, Robot Framework, JestJS, Cypress, Jasmine, Karma\\
		Cloud & Azure, Kubernetes, Helm, Docker\\
		CI/CD & GitlabCI, Jenkins, TravisCI
	\end{tabular}

\end{rSection}

\begin{rSection}{Projets}
	{\bf TeamMatesFinder - YNOV Campus}
	\\Plateforme permettant de rechercher des coéquipiers de jeux-vidéos sur des critères précis comme par exemple son niveau ou son rôle.
	\\ Technologies utilisées : Docker, PostgreSQL, Angular, ExpressJS, NodeJS.

	{\bf Boutique en ligne - AtelierCausseNature}
	\\Développement d'une boutique en ligne d'un artisant qui travaille avec le bois pour produire des bijoux ou des objets de décoration.
	\\ Technologies utilisées : NGINX, Symfony, PHP, MySQL, Bootstrap.

	{\bf Développement Android - YNOV Campus}
	\\Développement d'une application Android pour gérer l'emploi du temps d'un enfant, avec des tâches, réveils, alarmes et des rappels. 
	Elle permet de laisser de l'autonomie à l'enfant afin qu'il se responsabilise. Cette application est écrite en Java via Android Studio.
	\\ Technologies utilisées : Java - Android Studio, Firebase.

	{\bf Référent technique base de données - Aldostra}
	\\Conseils et gestion des bases de données du projet Aldostra, qui est un serveur Minecraft sur l'univers d'Harry Potter, plusieurs centaines de joueurs s'y connectent afin d'y incarner un personnage. J'ai également mit en place la migration de nos bases de données sous Redis vers MySQL en créant des scripts en Java pour transférer les milliers de données.
	\\ Technologies utilisées : Java, Redis, MySQL.

	{\bf Co-gérant du 1er forum francophone Skript - Skript-MC}
	\\Skrip-MC est le premier forum francophone Skript, qui est un plugin Minecraft permettant de simplifier le développement sur le jeu. 
	Notre communauté comporte près de 25 000 membres, je m'occupe de la gestion de l'équipe ainsi que le bon fonctionnement technique du forum.
	\\ Technologies utilisées : PHP, NGINX, XenForo, communication sur les réseaux sociaux, gestion d'équipe.

	{\bf DreamTech - Licence professionnelle}
	\\Développement d'une plateforme collaborative et ludique afin de mettre en relation des étudiants souhaitant réviser. 
	Il intégère un système de gamification pour motiver les utilisateurs à utiliser le site et ses fonctionnalités, comme un système d'expérience, de récompenses ou de classement. 
	Mise en place d'un système d'intégration continue grâce à TravisCI ainsi que des tests PHPUnit pour garantir une bonne maintenabilité du projet.
	\\ Technologies utilisées : Symfony, TravisCI, Bootstrap, JavaScript, MySQL, PHPUnit, Scrum.

\end{rSection}

\end{document}
